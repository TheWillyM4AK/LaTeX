\documentclass[                         %Clase de documento
12pt,                                   %Tamaño Letra/a4/
a4paper,                                %a4
oneside                                 %Páginas iguales
]{book}		                            %Libro

\usepackage[spanish,es-tabla]{babel}	%Paquete idioma, nombre para que a las tablas las llame tablas
\decimalpoint                           %Punto en vez de coma
\usepackage{bold-extra}                 %Para que no salte warning de fuentes por babel
\usepackage[utf8]{inputenc}				%Paquete para cambio de input (tildes)

\usepackage{graphicx}					%Paquete para poner imágenes y graficas en formato EPS
\usepackage{parskip}					%Paquete de gestion de parrafos

\usepackage{geometry}                   %Paquete para fijar los márgenes
\geometry{
a4paper,
left=2.5cm,
top=1cm,
right=2.5cm,
bottom=2cm,
headheight=2.5cm,
headsep=1cm,
foot=1cm,
footskip=1cm,
includeheadfoot
}

\usepackage[                            %Paquete para crear los hipervínculos
colorlinks=true,                        %activo colores en vez de cajas
urlcolor=blue,                          %link por defecto en rojo, cambio a azul url y citas
citecolor=blue
]{hyperref}

\usepackage[                            %Paquete para la bibliografía
backend=biber,                          %Compilador siempre bibar mas nuevo
style=numeric,                          %Estilo bibliografía
sorting=nyt                             %Orden
]{biblatex}
\usepackage{csquotes}                   %Para que no salte el warning, biblatex + babel

\usepackage[                            %Paquete para los acrónimos
acronym,
nonumberlist,                           %Sin contar las veces que salen
toc                                     %Salir en el índice
]{glossaries}

\usepackage{float}						%Paquete de flotantes para poner las figuras donde quiero
\usepackage{multirow}					%Paquete para juntar varias filas en las tablas
\usepackage[tableposition=top]{caption}	%Paquete para pones los títulos de las tablas arriba
\usepackage{array}                      %Paquete para poner m{} en las cols

\usepackage{fancyhdr}					%Paquete para encabezados y pie de página
\usepackage{lastpage}                   %Paquete poner en el pie la ultima pagina
\usepackage{titlesec}					%Paquete para poner section y subsectons al principio de la página
\usepackage{amsmath}					%Paquete matemático para poner letras griegas en negrita
\usepackage{textcomp}                   %Paquete para quitar warning en gensym, PONER ENCIMA
\usepackage{gensymb}                    %Paquete para poner º en math mode 
\usepackage{pdfpages}					%Paquete que me ha dado Sugon para meter la portada
\usepackage{afterpage}					%Paquete que me ha dado Sugon para meter la portada

 
\renewcommand{\baselinestretch}{1}		%Factor del interlineado
\setlength{\parskip}{\baselineskip}		%Tamaño de separación entre parrafos relativo al interlineado
\setlength{\parindent}{0cm}				%Indentación de parrafos a 0
\setcounter{secnumdepth}{3}				%Se numera hasta las subsecciones
\setcounter{tocdepth}{3}       			%Índice tiene encuenta hasta las subsubsecciones

%\newcommand{\sectionbreak}{\clearpage}	%Section en página nueva


\addbibresource{bibliografia.bib}		%Archivo .bib de donde coge la bibliografía

%Bloque de fancy para frontmatter distinto al mainmatter
\pagestyle{fancy}						%Defino mi estilo con el paquete fancy
\fancyhf{}								%Borro todo lo que viene por defecto, lo dejo limpio
%Encabezado con los logos
\fancyhead[R]{\includegraphics[height=2cm]{Logos/Logo_SATAA.png}}
\fancyhead[L]{\includegraphics[height=2cm]{Logos/Logo_UPM.png}}
%Pie de página
\fancyfoot[L]{Provisional}
\fancyfoot[R]{\thepage}
\renewcommand{\headrulewidth}{1pt}      % Grosor linea encabezado
\renewcommand{\footrulewidth}{1pt}      % Grosor linea pie
%Bloque de fancy para páginas de capítulo
\fancypagestyle{plain}{					%Para cambiar las páginas de capítulo
	\fancyhf{}							%limpio
	\fancyfoot[L]{Provisional}
    \fancyfoot[R]{\thepage}
	\renewcommand{\headrulewidth}{0pt}	%Quito la linea de arriba en los capitulos	
}

\makeglossaries
\newacronym{indicativo}{SIGLAS}{Nombre de las Siglas}
\glsaddall

\begin{document}

\begin{titlepage}
\centering
{\includegraphics[width=0.5\textwidth]{Logos/UPM}\par}
\vspace{0.1cm}
{\bfseries\LARGE Universidad Politécnica de Madrid \par}
\vspace{0.5cm}
{\scshape\large ESCUELA TÉCNICA SUPERIOR DE INGENIERÍA AERONÁUTICA Y DEL ESPACIO \par}
\vspace{0.25cm}
{\includegraphics[width=0.38\textwidth]{Logos/ETSIAE} \par}
\vspace{1cm}
{\scshape\Large Titulo \par}
\vspace{0.7cm}
{\itshape\large Asignatura \par}
\vfill
{\Large Autores: \par}
{\large Apellidos, Nombre \\ Apellidos, Nombre \\ Apellidos, Nombre \\ Apellidos, Nombre \par}
\vfill
{\Large día de mes de año \par}
\end{titlepage}

\frontmatter							%Todo lo contenido en frontmatter se numera en romano							
%Índice para que salga el resto de índices
{\hypersetup{linkcolor=black}	    	%Cambio el color de los link a negro
	\tableofcontents
	\cleardoublepage
	\addcontentsline{toc}{chapter}{\listfigurename}\listoffigures
	\cleardoublepage
	\addcontentsline{toc}{chapter}{\listtablename}\listoftables
}

\printglossary[type=\acronymtype,title=Acrónimos,toctitle=Acrónimos]

%Bloque de fancy para el resto del documento
\pagestyle{fancy}						%Defino mi estilo con el paquete fancy
\fancyhf{}								%Borro todo lo que viene por defecto, lo dejo limpio
%Encabezado con los logos
\fancyhead[R]{\includegraphics[height=2cm]{Logos/Logo_SATAA.png}}
\fancyhead[L]{\includegraphics[height=2cm]{Logos/Logo_UPM.png}}
%Pie de página
\fancyfoot[L]{Provisional}
\fancyfoot[R]{\thepage/{\hypersetup{linkcolor=black}\pageref{LastPage}}}
\renewcommand{\headrulewidth}{1pt}      % Grosor linea encabezado
\renewcommand{\footrulewidth}{1pt}      % Grosor linea pie

\mainmatter

\chapter{Capítulo}
\section{Sección}
\subsection{Subsección}
Texto
Cuerpo del documento

\backmatter

\nocite{*}	%Para que cite todo del .bib aunque no esté citado en el documento, HAY QUE PONERLO AL FINAL PORQUE METE TODAS LAS ENTRADAS DEL .BIB DE GOLPE Y ROMPE EL ORDEN. LAS FUENTES DEJAN DE ESTAR ORDENADAS POR ORDEN DE APARICIÓN, DEBE IR LO ÚLTIMO.
\printbibliography[heading=bibintoc,title={Bibliografía}]

\end{document}