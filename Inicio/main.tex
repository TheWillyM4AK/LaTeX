\documentclass[                         %Clase de documento
12pt,                                   %Tamaño Letra/a4/
a4paper,                                %a4
oneside                                 %Páginas iguales
]{book}		                            %Libro

\usepackage[spanish,es-tabla]{babel}	%Paquete idioma, nombre para que a las tablas las llame tablas
\decimalpoint                           %Punto en vez de coma
\usepackage{bold-extra}                 %Para que no salte warning de fuentes por babel
\usepackage[utf8]{inputenc}				%Paquete para cambio de input (tildes)

\usepackage{graphicx}					%Paquete para poner imágenes y graficas en formato EPS
\usepackage{parskip}					%Paquete de gestion de parrafos

\usepackage{geometry}                   %Paquete para fijar los márgenes
\geometry{
a4paper,
left=2.5cm,
top=1cm,
right=2.5cm,
bottom=2cm,
headheight=2.5cm,
headsep=1cm,
foot=1cm,
footskip=1cm,
includeheadfoot
}

\usepackage[                            %Paquete para crear los hipervínculos
colorlinks=true,                        %activo colores en vez de cajas
urlcolor=blue,                          %link por defecto en rojo, cambio a azul url y citas
citecolor=blue
]{hyperref}

 
\renewcommand{\baselinestretch}{1}		%Factor del interlineado
\setlength{\parskip}{\baselineskip}		%Tamaño de separación entre parrafos relativo al interlineado
\setlength{\parindent}{0cm}				%Indentación de parrafos a 0
\setcounter{secnumdepth}{3}				%Se numera hasta las subsecciones
\setcounter{tocdepth}{3}       			%Índice tiene encuenta hasta las subsubsecciones

\begin{document}

\begin{titlepage}
\centering
{\includegraphics[width=0.5\textwidth]{Logos/UPM}\par}
\vspace{0.1cm}
{\bfseries\LARGE Universidad Politécnica de Madrid \par}
\vspace{0.5cm}
{\scshape\large ESCUELA TÉCNICA SUPERIOR DE INGENIERÍA AERONÁUTICA Y DEL ESPACIO \par}
\vspace{0.25cm}
{\includegraphics[width=0.38\textwidth]{Logos/ETSIAE} \par}
\vspace{1cm}
{\scshape\Large Titulo \par}
\vspace{0.7cm}
{\itshape\large Asignatura \par}
\vfill
{\Large Autores: \par}
{\large Apellidos, Nombre \\ Apellidos, Nombre \\ Apellidos, Nombre \\ Apellidos, Nombre \par}
\vfill
{\Large día de mes de año \par}
\end{titlepage}

\frontmatter							%Todo lo contenido en frontmatter se numera en romano							
%Índice para que salga el resto de índices
{\hypersetup{linkcolor=black}	    	%Cambio el color de los link a negro
	\tableofcontents
	\cleardoublepage
	\addcontentsline{toc}{chapter}{\listfigurename}\listoffigures
	\cleardoublepage
	\addcontentsline{toc}{chapter}{\listtablename}\listoftables
}

\mainmatter

\include{}
\include{}
\include{}
\include{}

\backmatter

\end{document}